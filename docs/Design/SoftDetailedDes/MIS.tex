\documentclass[12pt, titlepage]{article}

\usepackage{amsmath, mathtools}

\usepackage[round]{natbib}
\usepackage{amsfonts}
\usepackage{amssymb}
\usepackage{graphicx}
\usepackage{colortbl}
\usepackage{xr}
\usepackage{hyperref}
\usepackage{longtable}
\usepackage{xfrac}
\usepackage{tabularx}
\usepackage{float}
\usepackage{siunitx}
\usepackage{booktabs}
\usepackage{multirow}
\usepackage[section]{placeins}
\usepackage{caption}
\usepackage{fullpage}
\usepackage{dsfont}

\hypersetup{
bookmarks=true,     % show bookmarks bar?
colorlinks=true,       % false: boxed links; true: colored links
linkcolor=red,          % color of internal links (change box color with linkbordercolor)
citecolor=blue,      % color of links to bibliography
filecolor=magenta,  % color of file links
urlcolor=cyan          % color of external links
}

\usepackage{array}

\externaldocument{../../SRS/SRS}

\input{../../Comments}
%% Common Parts

\newcommand{\progname}{Optimal EM Placement} % PUT YOUR PROGRAM NAME HERE
\newcommand{\authname}{Hussein Saad} % AUTHOR NAMES                  

\usepackage{hyperref}
    \hypersetup{colorlinks=true, linkcolor=blue, citecolor=blue, filecolor=blue,
                urlcolor=blue, unicode=false}
    \urlstyle{same}
                                


\begin{document}

\title{Module Interface Specification for \progname{}}

\author{\authname}

\date{\today}

\maketitle

\pagenumbering{roman}

\section{Revision History}

\begin{tabularx}{\textwidth}{p{3cm}p{2cm}X}
\toprule {\bf Date} & {\bf Version} & {\bf Notes}\\
\midrule
April 16, 2025 & 1.2 & Add moment and pose modules\\
\midrule
April 16, 2025 & 1.1 & Implement domain expert suggestions\\
\midrule
March 20, 2025 & 1.0 & Initial Release\\
\bottomrule
\end{tabularx}

~\newpage

\section{Symbols, Abbreviations and Acronyms}

See SRS Documentation at \url{https://github.com/husseinsd1/optimal-em-arrangement/blob/main/docs/SRS/SRS.pdf}
\newpage

\tableofcontents

\newpage

\pagenumbering{arabic}

\section{Introduction}

The following document details the Module Interface Specifications for
OEMP (Optimal Electromagnet Placement). This document describes, in detail, how the interfaces, assumptions and interactions among the modules of the program. 

Complementary documents include the \href{https://github.com/husseinsd1/optimal-em-arrangement/blob/main/docs/SRS/SRS.pdf}{System Requirement Specifications} and \href{https://github.com/husseinsd1/optimal-em-arrangement/blob/main/docs/Design/SoftArchitecture/MG.pdf}{Module Guide}. The full documentation and implementation can be
found at \url{https://github.com/husseinsd1/optimal-em-arrangement}.

\section{Notation}
The structure of the MIS for modules comes from \citet{HoffmanAndStrooper1995},
with the addition that template modules have been adapted from
\cite{GhezziEtAl2003}.  The mathematical notation comes from Chapter 3 of
\citet{HoffmanAndStrooper1995}.  For instance, the symbol := is used for a
multiple assignment statement and conditional rules follow the form $(c_1
\Rightarrow r_1 | c_2 \Rightarrow r_2 | ... | c_n \Rightarrow r_n )$.

The following table summarizes the primitive data types used by \progname. 

\begin{center}
\renewcommand{\arraystretch}{1.2}
\noindent 
\begin{tabular}{l l p{7.5cm}} 
\toprule 
\textbf{Data Type} & \textbf{Notation} & \textbf{Description}\\ 
\midrule
character & char & a single symbol or digit\\
integer & $\mathbb{Z}$ & a number without a fractional component in (-$\infty$, $\infty$) \\
natural number & $\mathbb{N}$ & a number without a fractional component in [1, $\infty$) \\
real & $\mathbb{R}$ & any number in (-$\infty$, $\infty$)\\
\bottomrule
\end{tabular} 
\end{center}

\noindent
The specification of \progname \ uses some derived data types: sequences, strings,
tuples, and vectors. Sequences are lists filled with elements of the same data type. Strings
are sequences of characters. Tuples contain a list of values, potentially of
different types. An $n$-dimensional vector is a list of $n$ real numbers. In addition, \progname \ uses functions, which
are defined by the data types of their inputs and outputs. Local functions are
described by giving their type signature followed by their specification.

\section{Module Decomposition}

The following table is taken directly from the Module Guide document for this project.

\begin{table}[h!]
\centering
\begin{tabular}{p{0.3\textwidth} p{0.6\textwidth}}
\toprule
\textbf{Level 1} & \textbf{Level 2}\\
\midrule

{Hardware-Hiding Module} & ~ \\
\midrule

\multirow{7}{0.3\textwidth}{Behaviour-Hiding Module} & Constant Parameters Module\\
& Input Parameters Module\\
& Magnetic Moment Module\\
& Magnetic Field Module\\
& Magnetic Force Module\\
& Generate Poses Module\\
& Actuation Matrix Module\\
& Output Results Module\\ 
& Main (Control) Module\\
\midrule

\multirow{3}{0.3\textwidth}{Software Decision Module} & \\
& Optimal Placement Module\\
\bottomrule

\end{tabular}
\caption{Module Hierarchy}
\label{TblMH}
\end{table}

\newpage
~\newpage

\section{MIS of Constant Parameters Module} \label{MISConsParam}
\subsection{Module}
ConstantParams

\subsection{Uses}
None

\subsection{Syntax}
\subsubsection{Exported Constants}
\begin{center}
\begin{tabular}{ c | c | c | c }
  Label & Symbol & Value & Description \\
  \hline
  MU0 & $\mu_0$ & $4\pi \times 10^{-7}$ & Permeability of free space
\end{tabular}
\end{center}

\subsubsection{Exported Access Programs}
None 

\subsection{Semantics}

\subsubsection{State Variables}
None

\subsubsection{Environment Variables}
None

\subsubsection{Assumptions}
Constant values are assumed immutable. 

\subsubsection{Access Routine Semantics}
None

\subsubsection{Local Functions}
None

%%%%%%%%%%%%%%%%%%%%%


\section{MIS of Input Parameters Module} \label{MISInpParam}

\subsection{Module}
params 

\subsection{Uses}
\begin{itemize}
  \item Hardware-Hiding Module
\end{itemize}

\subsection{Syntax}

\subsubsection{Exported Constants}
None 

\subsubsection{Exported Access Programs}
\begin{center}
\begin{tabular}{p{3cm} p{4cm} p{4cm} p{3cm}}
\hline
\textbf{Name} & \textbf{In} & \textbf{Out} & \textbf{Exceptions} \\
\hline
\texttt{takeInputs} & - & params of type \texttt{Params} & \texttt{negativeValueError}, \texttt{invalidCurrentError} \\
\hline
\end{tabular}
\end{center}

\subsection{Semantics}
\texttt{Params} is a data structure used to store the parameter values the user enters into the program. 
\subsubsection{State Variables}
params : \texttt{Params} $\:= $ [ \\ 
$N$ : $\mathbb{N}$, \\
$I$ : $\mathbb{R}$, \\
$A$ : $\mathbb{R}$, \\
$M$ : $\mathbb{N}$, \\
$K$ : $\mathbb{N}$, \\
$V$ : $\mathbb{R}$, \\
$t$ : $\mathbb{R}$, \\
$m_t$ : $\mathbb{R}$ \\
] \\
The description of the elements of the above array is found in Section 1.2 of the \href{https://github.com/husseinsd1/optimal-em-arrangement/blob/main/docs/SRS/SRS.pdf}{SRS}.

\subsubsection{Environment Variables}
\begin{itemize}
  \item A console: The medium through which the user will enter the parameter values. 
  \item A keyboard: The module takes input from the user's keyboard. 
\end{itemize}

\subsubsection{Assumptions}
None 

\subsubsection{Access Routine Semantics}

\noindent takeInputs():
\begin{itemize}
\item transition: 
\begin{itemize}
  \item Initialize params : \texttt{Params}. 
  \item Display prompt for user to enter the value of the parameters, one by one, and raise and exception and re-prompt if the entered value is not in line with the constraints outline in Table 2 of the \href{https://github.com/husseinsd1/optimal-em-arrangement/blob/main/docs/SRS/SRS.pdf}{SRS}. 
\end{itemize} 
\item output: params : \texttt{Params} 
\item exception: exc $\:=$
\begin{center}
  \begin{tabular}{p{5cm} p{4cm}}
  \hline
  \textbf{Exception} & \textbf{When} \\
  \hline
  \texttt{negativeValueError} & When the input for any of the parameters is negative.  \\
  \hline
  \texttt{invalidCurrentError} & When the input for the current $I$ is greater than 20000  \\
  \hline
  \end{tabular}
  \end{center}
\end{itemize}

\subsubsection{Local Functions}
None

\newpage
%%%%%%%%%%%%%%%%%%%%%%%%


\section{MIS of Magnetic Moment Module} \label{MISMagMoment}
\subsection{Module}
MagMoment

\subsection{Uses}
None

\subsection{Syntax}

\subsubsection{Exported Constants}
None

\subsubsection{Exported Access Programs}
\begin{center}
\begin{tabular}{p{3cm} p{4cm} p{4cm} p{3cm}}
\hline
\textbf{Name} & \textbf{In} & \textbf{Out} & \textbf{Exceptions} \\
\hline
calculateMoment & $N$ : $\mathbb{N}$, $I$ : $\mathbb{R}$, $A$ : $\mathbb{R}$  & in $\mathbb{R}$ & None \\
\hline
\end{tabular}
\end{center}

\subsection{Semantics}

\subsubsection{State Variables}
None

\subsubsection{Environment Variables}
None

\subsubsection{Assumptions}
None 

\subsubsection{Access Routine Semantics}

\noindent calculateMoment($N$ : $\mathbb{N}$, $I$ : $\mathbb{R}$, $A$ : $\mathbb{R}$):
\begin{itemize}
\item transition: N/A
\item output: The result of using the given inputs in the magnetic moment formula described in TM1 of the \href{https://github.com/husseinsd1/optimal-em-arrangement/blob/main/docs/SRS/SRS.pdf}{SRS}.
\item exception: N/A
\end{itemize}

\subsubsection{Local Functions}
None

\newpage
%%%%%%%%%%%%%%

\section{MIS of Magnetic Field Module} \label{MISMagField}
\subsection{Module}
MagField

\subsection{Uses}
\begin{itemize}
  \item Constant Parameters Module
  \item Input Parameters Module
\end{itemize}

\subsection{Syntax}

\subsubsection{Exported Constants}
None

\subsubsection{Exported Access Programs}
\begin{center}
\begin{tabular}{p{3cm} p{4cm} p{4cm} p{3cm}}
\hline
\textbf{Name} & \textbf{In} & \textbf{Out} & \textbf{Exceptions} \\
\hline
calculateField & params of type \texttt{Params} & vector in $\mathbb{R}^3$ & None \\
\hline
\end{tabular}
\end{center}

\subsection{Semantics}

\subsubsection{State Variables}
None

\subsubsection{Environment Variables}
None

\subsubsection{Assumptions}
Assumes the execution of the Input Parameters Module prior to running. 

\subsubsection{Access Routine Semantics}

\noindent calculateField(params : \texttt{Params}):
\begin{itemize}
\item transition:
\begin{itemize}
  \item Extract necessary parameters ($N, I, A, t$) from params. 
  \item Calculate and store magnetic moment using the calculateMoment() local function.
  \item Find the magnetic field with the given parameters, the calculated moment value, and $\mu_0$ from the Constant Parameters Module using the equation defined in TM2 of the \href{https://github.com/husseinsd1/optimal-em-arrangement/blob/main/docs/SRS/SRS.pdf}{SRS}. 
\end{itemize}
\item output: This module outputs a real 3D vector describing the magnetic field at some distance $t$ (retrieved from params).
\item exception: N/A
\end{itemize}

\subsubsection{Local Functions}
\textbf{calculateMoment(N : $\mathbb{N}$, I : $\mathbb{R}$, A : $\mathbb{R}$, t : $\mathbb{R}$):} A function to compute the magnetic moment at some distance $t$ as specified in TM1 of the \href{https://github.com/husseinsd1/optimal-em-arrangement/blob/main/docs/SRS/SRS.pdf}{SRS}.
\begin{itemize}
  \item output: A magnetic moment vector in $\mathbb{R}^3$
  \item exception: None
\end{itemize}

\newpage
%%%%%%%%%%%%%%%%

\section{MIS of Magnetic Force Module} \label{MISMagForce}

\subsection{Module}
MagForce 

\subsection{Uses}
\begin{itemize}
  \item Input Parameters Module 
  \item Magnetic Field Module
\end{itemize}

\subsection{Syntax}

\subsubsection{Exported Constants}
None 

\subsubsection{Exported Access Programs}
\begin{center}
  \begin{tabular}{p{3cm} p{4cm} p{4cm} p{3cm}}
  \hline
  \textbf{Name} & \textbf{In} & \textbf{Out} & \textbf{Exceptions} \\
  \hline
  calculateForce & params of type \texttt{Params} & vector in $\mathbb{R}^3$ & None \\
  \hline
  \end{tabular}
  \end{center}

\subsection{Semantics}

\subsubsection{State Variables}
None

\subsubsection{Environment Variables}
None

\subsubsection{Assumptions}
None 

\subsubsection{Access Routine Semantics}
\noindent calculateForce(params : \texttt{Params}):
\begin{itemize}
\item transition: 
\begin{itemize}
  \item Invoke and store the magnetic field returned by the Magnetic Field Module. 
  \item Extract the magnetic moment of the target object from params. 
  \item Compute the force vector as described in TM3 of the \href{https://github.com/husseinsd1/optimal-em-arrangement/blob/main/docs/SRS/SRS.pdf}{SRS}
\end{itemize}
\item output: A real 3D vector describing the magnetic force on some target. 
\item exception: N/A.
\end{itemize}

\subsubsection{Local Functions}
None
\newpage
%%%%%%%%%%

\section{MIS of Generate Poses Module} \label{MISGenPoses}
\subsection{Module}
GeneratePoses

\subsection{Uses}
None

\subsection{Syntax}

\subsubsection{Exported Constants}
None

\subsubsection{Exported Access Programs}
\begin{center}
\begin{tabular}{p{3cm} p{4cm} p{4cm} p{3cm}}
\hline
\textbf{Name} & \textbf{In} & \textbf{Out} & \textbf{Exceptions} \\
\hline
generatePoses & $M$ : $\mathbb{N}$, $l$ : $\mathbb{R}$ & array of $M$ pairs & None \\
\hline
\end{tabular}
\end{center}

\subsection{Semantics}

\subsubsection{State Variables}
None

\subsubsection{Environment Variables}
None

\subsubsection{Assumptions}
None 

\subsubsection{Access Routine Semantics}

\noindent generatePoses($M$ : $\mathbb{N}$, $l$ : $\mathbb{R}$):
\begin{itemize}
\item transition: N/A
\item output: An $M$ sized array of random poses, each generated with the \texttt{generatePose} function defined below.
\item exception: N/A
\end{itemize}

\subsubsection{Local Functions}
\textbf{generatePose}($l$ : $\mathbb{R}$): A function to generate a single random pose.
\begin{itemize}
  \item output: A random 3D coordinate within the under-the-table space, and a 3D rotation matrix.
  \item exception: None
\end{itemize}

\newpage
%%%%%%%%%%%%%%

\section{MIS of Actuation Matrix Module} \label{MISActMatrix}

\subsection{Module}
ActuationMatrix

\subsection{Uses}
\begin{itemize}
  \item Input Parameters Module
  \item Magnetic Field Module
  \item Magnetic Force Module 
\end{itemize}

\subsection{Syntax}

\subsubsection{Exported Constants}
None

\subsubsection{Exported Access Programs}
\begin{center}
\begin{tabular}{p{3cm} p{4cm} p{4cm} p{3cm}}
\hline
\textbf{Name} & \textbf{In} & \textbf{Out} & \textbf{Exceptions} \\
\hline
constructMatrix & params of type \texttt{Params} & in $\mathbb{R}^6$ & None \\
\hline
\end{tabular}
\end{center}

\subsection{Semantics}

\subsubsection{State Variables}
None

\subsubsection{Environment Variables}
None

\subsubsection{Assumptions}
None 

\subsubsection{Access Routine Semantics}

\noindent constructMatrix(params : \texttt{Params}):
\begin{itemize}
\item transition: 
\begin{itemize}
  \item Extract parameters from the params argument. 
  \item Generate a set of random positions (of size $M$ from params) using the local function generatePos.
  \item For each candidate position, calculate the magnetic force and field vectors using Magnetic Force Module and Magnetic Field Module, respectively. 
  \item Sum up the force and field vectors of the candidate positions, the result is two 3D vectors. 
  \item Concatenate the two vectors such that a $6 \times 1$ matrix is formed.
\end{itemize}
\item output: A $6 \times 1$ real matrix. 
\item exception: N/A
\end{itemize}

\subsubsection{Local Functions}
\textbf{generatePos(M : $\mathbb{N}$}): A function to generate random candidate positions for the EMs. 
\begin{itemize}
  \item output: A set of $M$ [x, y, z] coordinates. 
  \item exception: None
\end{itemize}

\newpage
%%%%%%%%%%


\section{MIS of Optimal Placement Module} \label{MISOptPlacement}

\subsection{Module}
FindOptPositions

\subsection{Uses}
\begin{itemize}
  \item Actuation Matrix Module 
  \item Input Parameters Module
\end{itemize}

\subsection{Syntax}

\subsubsection{Exported Constants}
None

\subsubsection{Exported Access Programs}

\begin{center}
  \begin{tabular}{p{3cm} p{4cm} p{4cm} p{3cm}}
  \hline
  \textbf{Name} & \textbf{In} & \textbf{Out} & \textbf{Exceptions} \\
  \hline
  solve & vector in $\mathbb{R}^6$ , params of type \texttt{Params} & binary vector in $\mathbb{R}^M$ & SolverException \\
  \hline
  \end{tabular}
  \end{center}

\subsection{Semantics}

\subsubsection{State Variables}
None

\subsubsection{Environment Variables}
None 

\subsubsection{Assumptions}
None

\subsubsection{Access Routine Semantics}

\noindent solve($\mathcal{U}$ : $\mathbb{R}^6$, params : \texttt{Params}):
\begin{itemize}
\item transition: 
\begin{itemize}
  \item Compute and store $\mathcal{U} \mathcal{U}^\top$ 
  \item Extract $M$ and $K$ from params.
  \item Pass $\mathcal{U} \mathcal{U}^\top$, $M$ and $K$ into a \texttt{cvxpy} solver.
\end{itemize}
\item output: A vector $x \in \{ 0, 1 \}^M$ such that:
\begin{itemize}
  \item $\mathds{1}_M^\top x = K$ ($\mathds{1}$ is a ones vector).
  \item $\lambda_{\text{min}}$ of $\sum^K_{i=1} x_i \mathcal{U}_i \mathcal{U}_i^\top$ is maximized. 
\end{itemize}
\item exception: Any exceptions raised by the solver.  
\end{itemize}

\subsubsection{Local Functions}
None

\newpage
%%%%%%%%%%%%%%%%

\section{MIS of Output Results Module} \label{MISResults}

\subsection{Module}
OutputResults 

\subsection{Uses}
\begin{itemize}
  \item Hardware-Hiding Module
  \item Optimal Placement Module 
\end{itemize}

\subsection{Syntax}

\subsubsection{Exported Constants}
None 

\subsubsection{Exported Access Programs}
\begin{center}
\begin{tabular}{p{3cm} p{4cm} p{4cm} p{3cm}}
\hline
\textbf{Name} & \textbf{In} & \textbf{Out} & \textbf{Exceptions} \\
\hline
output & $x \in \{0,1\}^M$  & - & None \\
\hline
\end{tabular}
\end{center}

\subsection{Semantics}

\subsubsection{State Variables}
None

\subsubsection{Environment Variables}
Console: this module prints the vector $x$ onto the console for the user to see.

\subsubsection{Assumptions}
None

\subsubsection{Access Routine Semantics}

\noindent output($x$ : $\{0,1\}^M$):
\begin{itemize}
\item transition: Prints the given vector onto the console. 
\item output: N/A 
\item exception: None
\end{itemize}

\subsubsection{Local Functions}
None 
\newpage
%%%%%%%%%%%%%%%%

\section{MIS of Main (Control) Module} \label{MISControl}

\subsection{Module}
main 

\subsection{Uses}
\begin{itemize}
  \item Hardware-Hiding Module 
  \item Constant Parameter Module 
  \item Input Parameters Module 
  \item Magnetic Field Module 
  \item Magnetic Force Module 
  \item Actuation Matrix Module 
  \item Optimal Placement Module 
  \item Output Results Module 
\end{itemize}

\subsection{Syntax}

\subsubsection{Exported Constants}
None 
\subsubsection{Exported Access Programs}
\begin{center}
\begin{tabular}{p{3cm} p{4cm} p{4cm} p{3cm}}
\hline
\textbf{Name} & \textbf{In} & \textbf{Out} & \textbf{Exceptions} \\
\hline
main & - & - & Various \\
\hline
\end{tabular}
\end{center}

\subsection{Semantics}

\subsubsection{State Variables}
\begin{itemize}
  \item params : \texttt{Params}
  \item $\mathcal{U}$ : $\mathbb{R}^6$
\end{itemize}

\subsubsection{Environment Variables}
None 

\subsubsection{Assumptions}
None 

\subsubsection{Access Routine Semantics}

\noindent main():
\begin{itemize}
\item transition: 
\begin{itemize}
  \item Call and store params from Input Parameters Module. 
  \item Invoke the Actuation Matrix Module and store the returned $\mathcal{U}$ vector (Actuation Matrix will itself invoke the modules responsible for the magnetic field/force and constant parameters).
  \item Provide the $\mathcal{U}$ vector and params to Optimal Placement Module, and store the returned $x$ vector. 
  \item Pass the returned $x$ vector to Output Results Module. 
\end{itemize}
\item output: N/A
\item exception: Exceptions arising from submodules.
\end{itemize}


\subsubsection{Local Functions}
None
%%%%%%%%%%%%

\newpage

\bibliographystyle {plainnat}
\bibliography {../../../refs/References}
\end{document}