\documentclass{article}

\usepackage{tabularx}
\usepackage{booktabs}

\title{Problem Statement and Goals\\\progname}

\author{\authname}

\date{}

%% Comments

\usepackage{color}

\newif\ifcomments\commentstrue %displays comments
%\newif\ifcomments\commentsfalse %so that comments do not display

\ifcomments
\newcommand{\authornote}[3]{\textcolor{#1}{[#3 ---#2]}}
\newcommand{\todo}[1]{\textcolor{red}{[TODO: #1]}}
\else
\newcommand{\authornote}[3]{}
\newcommand{\todo}[1]{}
\fi

\newcommand{\wss}[1]{\authornote{blue}{SS}{#1}} 
\newcommand{\plt}[1]{\authornote{magenta}{TPLT}{#1}} %For explanation of the template
\newcommand{\an}[1]{\authornote{cyan}{Author}{#1}}

%% Common Parts

\newcommand{\progname}{Optimal EM Placement} % PUT YOUR PROGRAM NAME HERE
\newcommand{\authname}{Hussein Saad} % AUTHOR NAMES                  

\usepackage{hyperref}
    \hypersetup{colorlinks=true, linkcolor=blue, citecolor=blue, filecolor=blue,
                urlcolor=blue, unicode=false}
    \urlstyle{same}
                                


\begin{document}

\maketitle

\begin{table}[hp]
\caption{Revision History} \label{TblRevisionHistory}
\begin{tabularx}{\textwidth}{llX}
\toprule
\textbf{Date} & \textbf{Developer(s)} & \textbf{Change}\\
\midrule
Jan.~17 & Hussein & Initial commit\\
\bottomrule
\end{tabularx}
\end{table}

\section{Problem Statement}

\subsection{Problem}
Microrobots are being explored in minimally invasive surgeries due to their ability to navigate millimeter spaces. The magnetic actuation systems used to control these microrobots suffer from bulkiness and can obstruct a surgeon's view. Electromagnetic actuators can also pose a safety risk due to the heat they emit as a result of the current they receive. 

\subsection{Proposed Solution}
We propose a design that places the actuation system under the operating table while only taking up limited space. We introduce optimization methods to solve the electromagnet (EM) arrangement problem. Central to our approach is an objective function — parameterized by the positions of the EMs — that is designed to maximize the minimum eigenvalue of the system’s \href{https://en.wikipedia.org/wiki/Manipulability_ellipsoid}{manipulability}, leading to an \href{https://en.wikipedia.org/wiki/Optimal_experimental_design}{E-optimal design}. 

\subsection{Inputs and Outputs}
\subsubsection{Inputs}
The program will receive the volume of the under-the-table workspace and properties of the EMs (\# of turns, area, current). In addition, the user will input their desired number of EMs in the system and the sample size (a larger size leads to a more optimal solution).

\subsubsection{Outputs}
The program will output the optimal positions of the EM actuators. 

\subsection{Stakeholders}
\subsubsection{General Stakeholders}
Surgeons and medical professionals that might operate an actuation system, engineers involved in the physical development of such systems, and researchers working on biomedical engineering systems and/or applications of optimization algorithms.

\subsubsection{Specialized Stakeholders}
Some specific stakeholders include Dr.~Onaizah (HeART Lab), and Dr.~Giamou (ARCO Lab).  

\subsection{Environment}
\subsubsection{Software}
The software is compatible with Windows, MacOS, and Linux.

\subsubsection{Hardware}
Any personal computer should work. 

\section{Goals}
\begin{itemize}
    \item The program will provide a physically feasible arrangement of electromagnets that can fit in an under-the-table workspace.
    \item The provided magnet arrangement will maximize the diversity of magnetic fields that can be generated, leading to increased control over the volume of the workspace. 
\end{itemize}

\section{Stretch Goals}
\begin{itemize}
    \item Accept different currents to independent EMs. 
    \item Allow for different kinds of EMs within the same workspace.
    \item Optimize individual EM angular configuration. 
\end{itemize}

\section{Challenge Level and Extras}
This project would be considered an \textbf{advanced} project as it is a continuation of research work submitted to ICRA 2025, and involves a novel approach to solving the magnet arrangement problem. 

\end{document}